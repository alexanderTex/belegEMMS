\documentclass[a4paper]{scrartcl}
\usepackage[ngerman]{babel}
\usepackage{amsmath,amssymb,amsthm,amsfonts,amsbsy,latexsym}
\usepackage[utf8]{inputenc}
\usepackage[T1]{fontenc}
\usepackage{enumerate,url}
\usepackage{graphicx}
\usepackage{bibgerm}
\usepackage[babel,german=guillemets]{csquotes}
%\usepackage{biblatex}
%\usepackage[style=numeric-comp]{biblatex}
\usepackage{listings}
\usepackage{color}
\usepackage[svgnames]{xcolor}
\usepackage{listings}		% Quelltext verwenden
\usepackage{color}
\usepackage{fancybox}  	% Box um Formel
\usepackage{varwidth}
\usepackage{lscape}
\usepackage{verbatim}	% mehrzeiliger Kommentar
\usepackage{titleref}	% Verweis auf Kapitel mit Titel

% /===========================================================================\
%
%   Anfang vom Dokument
%
% \===========================================================================/

\begin{document}
% /===========================================================================\
%
%   Titelseite
%
% \===========================================================================/
\thispagestyle{empty}

\begin{center}
\Large{Hochschule für Technik und Wirtschaft Berlin (HTW)}\\
\end{center}
 
 
\begin{center}
\Large{Fachbereich IV - Informatik, Kommunikation und Wirtschaft}
\end{center}
\begin{verbatim}


\end{verbatim}
\begin{center}
\textbf{\LARGE{Dokumentation}}
\end{center}
\begin{verbatim}
 
 
\end{verbatim}
\begin{center}
\textbf{im Studiengang Angewandte Informatik}
\end{center}
\begin{verbatim}
\end{verbatim}
 
\begin{flushleft}
\begin{tabular}{lll}
\textbf{Fach:} & & Entwicklung von Multimediaanwendungen\\
& & \\
& & \\
\textbf{Thema:} & & Sogo\\
\textbf{Bschreibung:}& & Viergewinnt im dreidimensionalen Raum (3D) \\
& & \\
& & \\
\textbf{eingereicht von:} & & Nils Brandt, 549906 \\
& & Alexander Lüdke, 548965 \\
& & \\
\textbf{eingereicht am:} & &  Sommersemester 2016 \\
& & \\
& & \\
\textbf{Dozenten:} & & Sebastian Bauer \\
& & 	Sebastian Keppler
\end{tabular}
\end{flushleft}

\newpage

% /===========================================================================\
%
%   Inhaltsverzeichnis
%
% \===========================================================================/
\thispagestyle{empty}

\tableofcontents

\newpage

% /===========================================================================\
%
%   Hauptdokument
%
% \===========================================================================/

\setcounter{page}{3}

% /===========================================================================\
%   Einleitung
% \===========================================================================/
\section{Einleitung}\label{Einleitung}
Die Applikation Sogo stellt den Komplettbeleg für das Fach Entwicklung von Multimediaanwendungen dar. Hierbei lag der Fokus sowohl auf der Funktionalität der Applikation als auch auf das Erarbeiten, Planen und Implementieren des Softwareprojektes. Dieses Projekt stellt zum gegenwärtigen Zeitpunkt, im vierten Semester des Studienfachs AI, das größte Softwareprojekt dar, welches als Beleg abzugeben ist.
\\
Die Grenzen der Realisierung sind weit gefasst, da sie zum Einen durch die Team-Mitglieder und zum Anderen durch die Dozenten mittels Anforderungen festgelegt wurden. Die Hauptimplementierung soll mit der Klassen-Bibliothek Qt, auf Basis von C++ erfolgen.

% /===========================================================================\
%   Grundlagen
% \===========================================================================/
\section{Grundlagen}\label{Grundlagen}
% Beschreibung des Spiels
Für unser Projekt haben wir uns für das Spiel Sogo entschieden, welches auch unter dem Namen Raummühle, 3D-Mühle oder Vier gewinnt Professional bekannt ist. Dabei gilt es wie auch im zweidimentsionalen Raum, seine Spielsteine direkt nebeneinander in der Horizontalen sowie in der Vertikalen zu platzieren und damit, je nach Spielfeldgröße, eine durchgehende Linie zu besetzten. In Sogo muss zusätzlich die dritte Dimension beachtet werden, in der es möglich ist, seine Seine diagonal und horizontal im Raum zu platzieren, was die Komplexität und damit den Schwierigkeitsgrad erhöht.
\\
Die Spielfeldgröße ist im Grundaufbau ein 4x4x4 Würfel, welchen wir für unsere Applikation auf 5x5x5 und 3x3x3 erweitert haben. 
Jeder Spieler verfügt im Raster
	\begin{itemize}
		\item 3x3x3 über 13 Spielsteine
		\item 4x4x4 über 32 Spielsteine		
		\item 5x5x5 über 62 Spielsteine
	\end{itemize}
Um sich innerhalb des Spielbrettes zu orientieren wird für die jeweilige Spielfeldgröße ein Raster festgelegt, welches sich wie folgt zusammensetzt. 
Für das Spielfeld
\begin{itemize}
		\item 3x3x3 wird die Grundfläche von ein bis neun und die Ebenen von 		eins bis drei durchnummeriert. Daraus folgt die Notation \\ $\mathcal{M}=\{(1,1),(1,2),(1,3),...(9,3)\}$.
		\item 4x4x4 gibt es die Möglichkeit die Grundfläche in Hexadezimal-Notation und die Ebenen von eins bis vier aufzuteilen. Daraus folgt die Notation \\
$\mathcal{M}=\{(1,1),(2,1),(3,1),...(F,4)\}$.
		\item 5x5x5 wird das bekannte Schachraster verwendet welches die Grundfläche von A-E und eins bis fünf sowie die die Ebenen von eins bis fünf festlegt. Daraus folgt die Notation \\
$\mathcal{M}=\{(A,1,1),(A,2,1),(A,3,1),...(E,5,5)\}$.
	\end{itemize}
	
Der grundlegende Spielverlauf ergibt sich wie folgt. Zuerst wird festgelegt, wer welche Spielsteinfarbe erhält und wer das Spiel beginnen darf. Anschließend stecken/setzen beide Spieler ihre Spielsteine auf den Stab der jeweiligen Position. Gewonnen hat der Spieler, der seine Spielsteine in einer Linie senkrecht, waagerecht, diagonal in einer Ebene oder diagonal nach oben bzw. unten gesetzt hat. Insgesamt sind 76 Möglichkeiten gegeben zu gewinnen, wobei auch ein unentschieden möglich ist. Dies geschieht, wenn alle Spielsteine von beiden Spielern gesetzt sind und keiner eine Linie mit seinen Spielsteinen besetzen konnte.

% /===========================================================================\
%   Analyse
% \===========================================================================/
\section{Analyse}\label{Analyse}

%Quellverweis - Lektüre
% Use-Case
Wie zuvor im Kapitel ~\ref{Einleitung} \glqq\titleref{Einleitung}\grqq \ auf Seite \pageref{Einleitung} beschrieben, sind die Grenzen der Realisierung sehr flexibel und mittels Anforderungen definiert, wobei wir in diesem Abschnitt nur auf die eingehen wollen, die unserer Meinung nach hervorzuheben sind und die von uns selber erfasst wurden.
\\
Die ersten Anforderungen waren schnell erfasst, weil wir uns an den Vorgaben der zur Verfügung gestellten Projekten orientierten, ohne zu wissen, wie der Aufwand einzuschätzen war, was Auswirkung auf das Ende des Projektes hatte. Doch dazu kommen wir im Abschnitt ~\ref{Ergebnis} \glqq\titleref{Ergebnis}\grqq \ auf Seite \pageref{Ergebnis}. Betrachten wir nun eine kleine Auswahl der zu implementierenden Anforderungen. 
\\
\\
\textbf{Anforderung:} 
Die Größe des Spielfelds soll vor der Partie in der GUI konfigurierbar sein. Die Standardgröße ist 4x4x4, wobei die Möglichkeit zum 3x3x3 oder 5x5x5 ausgewählt werden kann. 3x3x3 oder 5x5x5 ausgewählt werden kann.
\\
\textbf{Beschreibung:} Das Spielfeld stellte die Grundlage für Sogo dar. Natürlich sollte es möglich sein die Spielfeldgröße auszuwählen. Als wir uns überlegten wie wir das Spielfeld organisieren sollten, kam uns der Gedanke das 5x5x5 Spielfeld variabel zu gestalten bzw. die Steine auf eine andere Art zu setzen. Wir konzipierten das Spielfeld so, dass die Grundfläche, nicht wie bei 4x4x4 und 3x3x3 unten liegt, sonder in der mittleren Schicht liegt also auf Ebene drei.  Dadurch war der Spieler in der Lage seine Steine sowohl von oben als auch von unten zu setzen. Nachdem wir im weiteren Verlauf der Entwicklung festgestellt hatten, dass die Implementierung des einfachen Spielfeldes schon sehr viel Zeit in Anspruch nahm, mussten wir diese Spieloption vorerst zurückstellen. Was dazu führte das in der Version 1.0 das Spielfeld 5x5x5 in der Standardkonfiguration gespielt werden kann.
\\ 
\\
\textbf{Anforderung:} 
Die aktuelle Spielhistorie wird für das laufende Spiel angezeigt.
\\ 
\textbf{Beschreibung:} Dies war uns wichtig, de es dem Spieler die Möglichkeit geben sollte, die bereits gespielten Züge nachvollziehen zu könne und sie bei Gelegenheit zu analysieren.
\\
\\
\textbf{Anforderung:} 
Beim Spiel Mensch gegen Computer soll es möglich sein, die Spielstärke des Computer aus drei Schwierigkeitsgraden auszuwählen.
\\
\textbf{Beschreibung:} Hierbei war ein Hauch von künstlicher Intelligenz zu erahnen, welchen wir nutzten, um den Minimax-Algorithmus zu implementieren, der die optimale Spielstrategie der AI festlegen sollte. Hierzu betrachten wir im Kapitel ~\ref{AI} \glqq\titleref{AI}\grqq \ auf Seite \pageref{AI} den Quellcode der Implementierung.
\\
\\
\textbf{Anforderung:} 
Das Spiel ist sowohl in 2D- als auch in 3D-Ansicht spielbar/visualisierbar.
\\
\textbf{Beschreibung:} Diese Anforderung stellte einen immensen Arbeitsaufwand dar, der zuvor nicht einzuschätzen war. Hierzu müssen wir etwas weiter ausholen, da diese Anforderung nicht nur das Studienfach Entwicklung von Multimediaanwendungen sondern auch das Fach Computergrafik betraf. Wir hatten vor beide Fächer in einem Projekt zu vereinen. Im Fach Computergrafik mussten wir mit der Grafikbibliothek OpenGL(Open Graphics Library) eine 3D-Szene erstellen. Diese Szene sollte sowohl in 3D als auch interaktiv sein. Innerhalb der Vorlesung erstellten wir unter Visual Studio 2015 eine lauffähige Version von Sogo in 3D, was im Anschluss folgte, war \glqq lediglich\grqq \ die Überführung und Herstellung der Lauffähigkeit der 3D Ansicht in Verbindung mit einer bis dahin erstellten Spiellogik. Letztendlich brauchten wir, nur für die  Implementierung dieser Anforderung, vier Wochen. Was zum Einen mit der Überführung von OpenGL-Syntax in die Qt-Syntax zu tun hatte und zum Andern mit der OpenGL Version der einzelnen Systeme, welche für die Erstellung zum Einsatz kamen. Im Labor war es nur möglich unter Ubuntu Qt zu nutzen, was jedoch bedeutete eine alte Version von OpenGL zu verwenden, die mit der Implementierungsversion nicht kompatibel war. Die einzige Möglichkeit war, die komplette Umsetzung der 3D-Umgebung auf den heimischen Rechner zu realisieren.
\\ 
\\
Nachdem wir die Anforderungen ausformuliert hatten, legten wir uns ein Git-Repository an, welches wir für das Projekt verwendeten. Dieses wurde anfänglich genutzt um gemeinsame Dokumente wie den Anforderungsbericht und einen zentralen ToDo-Zettel zu verwalten, der als Projekttagebuch dienen sollte. In diesem Projekttagebuch legten wir fest wer welche Anforderung übernahm, wie der Status des Bearbeitung ist und welche zusätzlichen Arbeiten noch zu erledigen waren, die vorher nicht ersichtlich waren. Darüber hinaus nutzten wir ihn, um uns gegenseitig Zusatzinformation zukommen zu lassen, die uns während der eigenen Bearbeitung auffielen und die wir dem jeweils anderen mitteilen wollten. Der Hauptverwendungszweck lag definitiv auf der Versionierung unseres Quelltextes.

% /===========================================================================\
%   Entwurf
% \===========================================================================/
\section{Entwurf}\label{Entwurf}

% Klassendiagramm

% /===========================================================================\
%   Implementation
% \===========================================================================/
\section{Implementierung}\label{Implementierung}

% Das Raster der jeweiligen Spielfelder wurde vereinfacht auf Grund einfacher Physik, die Schwekraft. Daher ergibt sich für jedes Spiel feld eine Notation von (1,1)... (n,m).

% Verwendung von Code-Snipets

% core/
\subsection{Vector2}\label{Vector2}
\subsection{Vector3}\label{Vector3}
\subsection{Artificial Intelligence(AI)}\label{AI}
\subsection{PlayingField}\label{PlayingField}
\subsection{GameData}\label{GameData}
\subsection{Player}\label{Player}


% gui/
\subsection{Spielmenüs}\label{Spielmenüs}
\subsubsection{StartMenu}\label{StartMenu}
\subsubsection{NewSessionMenu}\label{NewSessionMenu}
\subsubsection{HighscoreMenu}\label{HighscoreMenu}
\subsubsection{PauseMenu}\label{PauseMenu}

\subsection{Mainwindow}\label{Mainwindow}
\subsection{GameView}\label{GameView}
\subsubsection{GameView2D}\label{GameView2D}
\subsubsection{GameView3D}\label{GameView3D}
\subsection{GameVisualizer}\label{GameVisualizer}
\subsection{Playerinput}\label{Playerinput}
\subsection{HistoryDisplay}\label{HistoryDisplay}


%utility
\subsubsection{Logger - Der hauseigene Debugger}\label{Logger}

%external
\subsection{3D Umsetzung}\label{3DUmsetzung}

% /===========================================================================\
%   Test
% \===========================================================================/
\section{Test}\label{Test}
\subsection{Unit-Test}\label{Unit}
\subsection{Valgrind}\label{Valgrind}

% /===========================================================================\
%   Ergebnis
% \===========================================================================/
\section{Ergebnis}\label{Ergebnis}
% Auswertung
\subsection{Vergleich}\label{Vergleich}
\subsection{Zusammenfassung}\label{Zusammenfassung}
\subsection{Bewertung}\label{Bewertung}
\subsection{Ausblick}\label{Ausblick}

\end{document}